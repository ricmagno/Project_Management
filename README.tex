% Created 2022-03-21 Mon 18:54
% Intended LaTeX compiler: pdflatex
\documentclass[11pt]{article}
\usepackage[utf8]{inputenc}
\usepackage[T1]{fontenc}
\usepackage{graphicx}
\usepackage{longtable}
\usepackage{wrapfig}
\usepackage{rotating}
\usepackage[normalem]{ulem}
\usepackage{amsmath}
\usepackage{amssymb}
\usepackage{capt-of}
\usepackage{hyperref}
\usepackage[margin=2cm]{geometry}
\author{Ricardo Antunes}
\date{\today}
\title{}
\hypersetup{
 pdfauthor={Ricardo Antunes},
 pdftitle={},
 pdfkeywords={},
 pdfsubject={},
 pdfcreator={Emacs 27.2 (Org mode 9.4.6)}, 
 pdflang={English}}
\begin{document}

\tableofcontents

\section{Project Management}
\label{sec:org456a8f7}
Project Management org templates

\subsection{STRATEGY ARTIFACTS}
\label{sec:org8fabcd0}
Documents that are created prior to or at the start of the project that address strategic, business, or high-level information about the project. Strategy artifacts are developed at the start of a project and do not normally change, though they may be reviewed throughout the project.
\begin{itemize}
\item Business case. A business case is a value proposition for a proposed project that may include financial and nonfinancial benefits.
\item Business model canvas. This artifact is a one-page visual summary that describes the value proposition, infrastructure, customers, and finances. These are often used in lean start-up situations.
\item Project brief. A project brief provides a high-level overview of the goals, deliverables, and processes for the project.
\item Project charter. A project charter is a document issued by the project initiator or sponsor that formally authorizes the existence of a project and provides the project manager with the authority to apply organizational resources to project activities.
\item Project vision statement. This document is a concise, high-level description of the project that states the purpose, and inspires the project team to contribute to the project.
\item Roadmap. This document provides a high-level time line that depicts milestones, significant events, reviews, and decision points.
\end{itemize}

\subsection{LOGS AND REGISTERS}
\label{sec:org9e53622}
Logs and registers are used to record continuously evolving aspects of the project. They are updated throughout the project. The terms log and register are sometimes used interchangeably. It is not uncommon to see the term risk register or risk log referring to the same artifact.
\begin{itemize}
\item Assumption log. An assumption is a factor that is considered to be true, real, or certain, without proof or demonstration. A constraint is a factor that limits the options for managing a project, program, portfolio, or process. An assumption log records all assumptions and constraints throughout the project.
\item Backlog. A backlog is an ordered list of work to be done. Projects may have a product backlog, a requirements backlog, impediments backlog, and so forth. Items in a backlog are prioritized. The prioritized work is then scheduled for upcoming iterations.
\item Change log. A change log is a comprehensive list of changes submitted during the project and their current status. A change can be a modification to any formally controlled deliverable, project management plan component, or project document.
\item Issue log. An issue is a current condition or situation that may have an impact on the project objectives. An issue log is used to record and monitor information on active issues. Issues are assigned to a responsible party for follow up and resolution.
\item Lessons learned register. A lessons learned register is used to record knowledge gained during a project, phase, or iteration so that it can be used to improve future performance for the project team and/or the organization.
\item Risk-adjusted backlog. A risk-adjusted backlog is a backlog that includes work and actions to address threats and opportunities.
\item Risk register. A risk register is a repository in which outputs of risk management processes are recorded. Information in a risk register can include the person responsible for managing the risk, probability, impact, risk score, planned risk responses, and other information used to get a high-level understanding of individual risks.
\item Stakeholder register. A stakeholder register records information about project stakeholders, which includes an assessment and classification of project stakeholders.
\end{itemize}

\subsection{PLANS}
\label{sec:orgbe8ecfb}
A plan is a proposed means of accomplishing something. Project teams develop plans for individual aspects of a project and/or combine all of that information into an overarching project management plan. Plans generally are written documents but may also be reflected on visual/ virtual whiteboards.
\begin{itemize}
\item Change control plan. A change control plan is a component of the project management plan that establishes the change control board, documents the extent of its authority, and describes how the change control system will be implemented.
\item Communications management plan. This plan is a component of the project, program, or portfolio management plan that describes how, when, and by whom information about the project will be administered and disseminated.
\item Cost management plan. This plan is a component of a project or program management plan that describes how costs will be planned, structured, and controlled.
\item Iteration plan. This plan is a detailed plan for the current iteration.
\item Procurement management plan. This plan is a component of the project or program management plan that describes how a project team will acquire goods and services from outside of the performing organization.
\item Project management plan. The project management plan is a document that describes how the project will be executed, monitored and controlled, and closed.
\item Quality management plan. This plan is a component of the project or program management plan that describes how applicable policies, procedures, and guidelines will be implemented to achieve the quality objectives.
\item Release plan. This plan sets expectations for the dates, features, and/or outcomes expected to be delivered over the course of multiple iterations.
\item Requirements management plan. This plan is a component of the project or program management plan that describes how requirements will be analyzed, documented,and managed.
\item Resource management plan. This plan is a component of the project management plan that describes how project resources are acquired, allocated, monitored, and controlled.
\item Risk management plan. This plan is a component of the project, program, or portfolio management plan that describes how risk management activities will be structuredand performed.
\item Scope management plan. This plan is a component of the project or program management plan that describes how the scope will be defined, developed, monitored, controlled, and validated.
\item Schedule management plan. This plan is a component of the project or program management plan that establishes the criteria and the activities for developing, monitoring, and controlling the schedule.
\item Stakeholder engagement plan. This plan is a component of the project management plan that identifies the strategies and actions required to promote productive involvement of stakeholders in project or program decision making and execution.
\item Test plan. This document describes deliverables that will be tested, tests that will be conducted, and the processes that will be used in testing. It forms the basis for formally testing the components and deliverables.
\end{itemize}

\subsection{HIERARCHY CHARTS}
\label{sec:orge16ecb0}
Hierarchy charts begin with high-level information that is progressively decomposed into greater levels of detail. The information at the upper levels encompasses all the information at the lower or subsidiary levels. Hierarchy charts are often progressively elaborated into greater levels of detail as more information is known about the project.
\begin{itemize}
\item Organizational breakdown structure. This chart is a hierarchical representation of the project organization, which illustrates the relationship between project activities and the organizational units that will perform those activities.
\item Product breakdown structure. This chart is a hierarchical structure reflecting a product’s components and deliverables.
\item Resource breakdown structure. This chart is a hierarchical representation of resources by category and type.
\item Risk breakdown structure. This chart is a hierarchical representation of potential sources of risks.
\item Work breakdown structure. This chart is a hierarchical decomposition of the total scope of work to be carried out by the project team to accomplish the project objectives and create the required deliverables.
\end{itemize}

\subsection{BASELINES}
\label{sec:org7db18cb}
A baseline is the approved version of a work product or plan. Actual performance is compared to baselines to identify variances.
\begin{itemize}
\item Budget. A budget is the approved estimate for the project or any work breakdown structure (WBS) component or any schedule activity.
\item Milestone schedule. This type of schedule presents milestones with planned dates.
\item Performance measurement baseline. Integrated scope, schedule, and cost baselines
\end{itemize}
are used for comparison to manage, measure, and control project execution.
\begin{itemize}
\item Project schedule. A project schedule is an output of a schedule model that presents linked activities with planned dates, durations, milestones, and resources.
\item Scope baseline. This baseline is the approved version of a scope statement, work breakdown structure (WBS), and its associated WBS dictionary that can be changed using formal change control procedures and is used as the basis for comparison to actual results.
\end{itemize}

\subsection{VISUAL DATA AND INFORMATION}
\label{sec:org259c160}
Visual data and information are artifacts that organize and present data and information in a visual format, such as charts, graphs, matrices, and diagrams. Visualizing data makes it easier to absorb data and turn it into information. Visualization artifacts are often produced after data have been collected and analyzed. These artifacts can aid in decision making and prioritization.
\begin{itemize}
\item Affinity diagram. This diagram shows large numbers of ideas classified into groups for review and analysis.
\item Burndown/burnup chart. This chart is a graphical representation of the work remaining in a timebox or the work completed toward the release of a product or project deliverable.
\item Cause-and-effect diagram. This diagram is a visual representation that helps trace an undesirable effect back to its root cause.
\item Cumulative flow diagram (CFD). This chart indicates features completed over time, features in development, and those in the backlog. It may also include features at intermediate states, such as features designed but not yet constructed, those in quality assurance, or those in testing.
\item Cycle time chart. This diagram shows the average cycle time of the work items completed over time. A cycle time chart may be shown as a scatter diagram or a bar chart.
\item Dashboards. This set of charts and graphs shows progress or performance against important measures of the project.
\item Flowchart. This diagram depicts the inputs, process actions, and outputs of one or more processes within a system.
\item Gantt chart. This bar chart provides schedule information where activities are listed on the vertical axis, dates are shown on the horizontal axis, and activity durations are shown
\end{itemize}
as horizontal bars placed according to start and finish dates.
\begin{itemize}
\item Histogram. This bar chart shows the graphical representation of numerical data.
\item Information radiator. This artifact is a visible, physical display that provides information
\end{itemize}
to the rest of the organization, enabling timely knowledge sharing.
\begin{itemize}
\item Lead time chart. This diagram shows the trend over time of the average lead time of the items completed in work. A lead time chart may be shown as a scatter diagram or a bar chart.
\item Prioritization matrix. This matrix is a scatter diagram where effort is shown on the horizontal axis and value on the vertical axis, divided into four quadrants to classify items by priority.
\item Project schedule network diagram. This graphical representation shows the logical relationships among the project schedule activities.
\item Requirements traceability matrix. This matrix links product requirements from their origin to the deliverables that satisfy them.
\item Responsibility assignment matrix (RAM). This matrix is a grid that shows the project resources assigned to each work package. A RACI chart is a common way of showing stakeholders who are responsible, accountable, consulted, or informed and are associated with project activities, decisions, and deliverables.
\item Scatter diagram. This graph shows the relationship between two variables.
\item S-curve. This graph displays cumulative costs over a specified period of time.
\item Stakeholder engagement assessment matrix. This matrix compares current and desired stakeholder engagement levels.
\item Story map. A story map is a visual model of all the features and functionality desired for a given product, created to give the project team a holistic view of what they are building and why.
\item Throughput chart. This chart shows the accepted deliverables over time. A throughput chart may be shown as a scatter diagram or a bar chart.
\item Use case. This artifact describes and explores how a user interacts with a system to achieve a specific goal.
\item Value stream map. This is a lean enterprise method used to document, analyze, and improve the flow of information or materials required to produce a product or service for a customer. Value stream maps can be used to identify waste.
\item Velocity chart. This chart tracks the rate at which the deliverables are produced, validated, and accepted within a predefined interval.
\end{itemize}

\subsection{REPORTS}
\label{sec:org15493c1}
Reports are formal records or summaries of information. Reports communicate relevant (usually summary level) information to stakeholders. Often reports are given to stakeholders who are interested in the project status, such as sponsors, business owners, or PMOs.
\begin{itemize}
\item Quality report. This project document includes quality management issues, recommendations for corrective actions, and a summary of findings from quality control activities. It may include recommendations for process, project, and product improvements.
\item Risk report. This project document is developed progressively throughout the risk management processes and summarizes information on individual project risks and the level of overall project risk.
\item Status report. This document provides a report on the current status of the project. It may include information on progress since the last report and forecasts for cost and schedule performance.
\end{itemize}

\subsection{AGREEMENTS AND CONTRACTS}
\label{sec:org9cf030b}
An agreement is any document or communication that defines the intentions of the parties. In projects, agreements take the form of contracts or other defined understandings. A contract is
a mutually binding agreement that obligates the seller to provide the specified product, service, or result and obligates the buyer to pay for it. There are different types of contracts, some of which fall within a category of fixed-price or cost-reimbursable contracts.
\begin{itemize}
\item Fixed-price contracts. This category of contract involves setting a fixed price for a well-defined product, service, or result. Fixed-price contracts include firm fixed price (FFP), fixed-price incentive fee (FPIF), and fixed price with economic price adjustment (FP-EPA), among others.
\item Cost-reimbursable contracts. This category of contracts involves payments to the seller for actual costs incurred for completing the work plus a fee representing seller profit. These contracts are often used when the project scope is not well defined or is subject to frequent change. Cost-reimbursable contracts include cost plus award fee (CPAF), cost plus fixed fee (CPFF), and cost plus incentive fee (CPIF).
\item Time and materials (T\&M). This contract establishes a fixed rate, but not a precise statement of work. It can be used for staff augmentation, subject matter expertise, or other outside support.
\item Indefinite delivery indefinite quantity (IDIQ). This contract provides for an indefinite quantity of goods or services, with a stated lower and upper limit, and within a fixed
\end{itemize}
time period. These contracts can be used for architectural, engineering, or information technology engagements.
\begin{itemize}
\item Other agreements. Other types of agreements include memorandum of understanding (MOU), memorandum of agreement (MOA), service level agreement (SLA), basic ordering agreement (BOA), among others.
\end{itemize}

\subsection{OTHER ARTIFACTS}
\label{sec:orgde3b51f}
The documents and deliverables described here do not fit into a specific category; however, they are important artifacts that are used for a variety of purposes.
\begin{itemize}
\item Activity list. This document provides a tabulation of schedule activities that shows the activity description, activity identifier, and a sufficiently detailed scope of work description so project team members understand what work is to be performed.
\item Bid documents. Bid documents are used to request proposals from prospective sellers. Depending on the goods or services needed, bid documents can include, among others:
\begin{itemize}
\item Request for information (RFI),
\item Request for quotation (RFQ), and
\item Request for proposal (RFP).
\end{itemize}
\item Metrics. Metrics describe an attribute and how to measure it.
\item Project calendar. This calendar identifies working days and shifts that are available
\end{itemize}
for scheduled activities.
\begin{itemize}
\item Requirements documentation. This document is a record of product requirements and relevant information needed to manage the requirements, which includes the associated category, priority, and acceptance criteria.
\item Project team charter. This document records the project team values, agreements,
\end{itemize}
and operating guidelines, and establishes clear expectations regarding acceptable behavior by project team members.
\begin{itemize}
\item User story. A user story is a brief description of an outcome for a specific user, which is a promise of a conversation to clarify details.
\end{itemize}

\subsection{ARTIFACTS APPLIED ACROSS PERFORMANCE DOMAINS}
\label{sec:org1ff3fdd}
Different artifacts are more likely to be useful in different performance domains. While the delivery approach, product, and organizational environment will determine which artifacts are most applicable for a specific project, there are some performance domains that are more likely to make use of specific artifacts. Table 4-3 suggests the performance domain(s) where each artifact is more likely to be of use; however, the project manager and/or project team has the ultimate responsibility for selecting and tailoring the artifacts for their project.


\label{artifacts_mapping}
\begin{center}
\begin{tabular}{lllllllll}
Artifact & Team\footnotemark & Stake\footnotemark & DA LC \footnotemark & Plan\footnotemark & P Work\footnotemark & Deliv\footnotemark & Measu\footnotemark & Uncer\footnotemark\\
\hline
\textbf{Strategy Artifacts:} &  &  &  &  &  &  &  & \\
\hline
Business case &  & x &  & x &  &  &  & \\
Project brief &  & x &  & x &  &  &  & \\
Project charter &  & x &  & x &  &  &  & \\
Project vision statement &  & x &  & x &  &  &  & \\
Roadmap &  & x & x & x &  &  &  & \\
\textbf{Log and Register Artifacts:} &  &  &  &  &  &  &  & \\
Assumption log &  &  &  & x & x & x &  & x\\
Backlog &  &  &  & x & x & x &  & \\
Change log &  &  &  &  & x & x &  & \\
Issue log &  &  &  &  & x &  &  & \\
Lessons learned register &  &  &  &  & x &  &  & \\
Risk-adjusted backlog &  &  &  & x &  &  &  & x\\
Risk register &  &  &  & x & x & x &  & x\\
Stakeholder register &  & x &  & x &  &  &  & \\
\hline
\textbf{Plan Artifacts:} &  &  &  & x & x & x &  & \\
\hline
Change control plan &  & x &  & x & x &  &  & \\
Communications management plan &  &  &  & x &  &  &  & \\
Cost management plan &  &  &  & x &  &  &  & \\
Iteration plan &  &  &  & x &  &  &  & \\
Procurement management plan &  &  &  & x & x &  &  & \\
Project management plan &  & x &  & x & x &  &  & \\
Quality management plan &  &  &  & x & x &  &  & \\
Release plan &  &  &  & x &  & x &  & \\
Requirements management plan &  &  &  & x &  & x &  & \\
Resource management plan &  &  &  & x & x &  &  & \\
Risk management plan &  &  &  & x & x &  &  & x\\
Scope management plan &  &  &  & x &  & x &  & \\
Schedule management plan &  &  &  & x & x & x &  & \\
Stakeholder engagement plan &  & x &  & x &  &  &  & \\
Test plan &  &  &  & x & x & x & x & \\
\hline
\textbf{Hierarchy Chart Artifacts:} &  &  &  &  &  &  &  & \\
\hline
Organizational breakdown structure & x & x &  & x &  &  &  & \\
Product breakdown structure &  &  &  & x &  & x &  & \\
Resource breakdown structure & x &  &  & x & x &  & x & \\
Risk breakdown structure &  &  &  &  & x &  &  & x\\
Work breakdown structure &  &  &  & x &  & x & x & \\
\hline
\textbf{Baseline Artifacts:} &  &  &  &  &  &  &  & \\
\hline
Budget &  &  &  & x & x &  & x & \\
Milestone schedule &  &  & x & x & x &  & x & \\
Performance measurement baseline &  &  &  & x & x & x & x & \\
Project schedule &  &  &  & x & x &  & x & \\
Scope baseline &  &  &  & x & x &  & x & \\
\hline
\textbf{Visual Data and Information Artifacts:} &  &  &  & x & x & x & x & \\
\hline
Affinity diagram &  &  &  & x & x &  & x & \\
Burn chart &  &  &  & x &  & x & x & \\
Cause-and-effect diagram &  &  &  &  & x & x &  & x\\
Cycle time chart &  &  &  &  &  & x & x & \\
Cumulative flow diagram &  &  &  &  &  & x & x & \\
Dashboard &  &  &  &  & x &  & x & \\
Flow chart &  &  &  &  & x & x & x & \\
Gantt chart &  &  &  & x & x &  & x & \\
Histogram &  &  &  &  &  &  & x & \\
Information radiator &  &  &  &  & x &  & x & \\
Lead time chart &  &  &  &  &  & x & x & \\
Prioritization matrix &  & x &  &  & x & x &  & \\
Project schedule network diagram &  &  &  & x & x &  &  & \\
Requirements traceability matrix &  &  &  & x &  & x & x & \\
Responsibility assignment matrix &  &  &  & x & x &  &  & \\
Scatter diagram &  &  &  &  & x & x & x & \\
S-curve &  &  &  & x &  &  & x & \\
Stakeholder engagement assessment matrix &  & x &  & x & x &  &  & \\
Story map &  &  &  & x &  & x &  & \\
Throughput chart &  &  &  &  &  & x & x & \\
Use case &  &  &  & x &  & x &  & \\
Value stream map &  &  &  &  & x & x & x & \\
Velocity chart &  &  &  &  &  & x & x & \\
\hline
\textbf{Report Artifacts:} &  &  &  &  &  &  &  & \\
\hline
Quality report &  &  &  &  & x & x & x & \\
Risk report &  &  &  &  & x &  &  & x\\
Status report &  &  &  &  & x &  &  & \\
\hline
\textbf{Agreements and Contracts:} &  &  &  &  &  &  &  & \\
\hline
Fixed-price &  & x &  & x & x & x & x & x\\
Cost-reimbursable &  & x &  & x & x & x & x & x\\
Time and materials &  & x &  & x & x & x & x & x\\
Indefinite time indefinite quantity (IDIQ) &  & x &  & x & x & x & x & x\\
Other agreements &  & x &  & x & x & x & x & x\\
\hline
\textbf{Other Artifacts:} &  &  &  &  &  &  &  & \\
\hline
Activity list & x & x &  & x & x &  &  & \\
Bid documents &  & x &  & x & x &  &  & \\
Metrics &  &  &  & x &  & x & x & \\
Project calendars & x &  &  & x & x &  &  & \\
Requirements documentation &  & x &  & x &  & x & x & \\
Project team charter & x &  &  & x &  &  &  & \\
User story &  & x &  & x &  & x &  & \\
\end{tabular}
\end{center}\footnotetext[1]{\label{org440aec5}Team}\footnotetext[2]{\label{org01dd5e4}Stakeholders}\footnotetext[3]{\label{orgf756d6e}Dev Approach and Life Cycle}\footnotetext[4]{\label{org9d5187e}Planning}\footnotetext[5]{\label{org40c10ee}Project Work}\footnotetext[6]{\label{org3d89f23}Delivery}\footnotetext[7]{\label{orgaa9efe2}Measurement}\footnotetext[8]{\label{orgd5fe94e}Uncertainty}
\end{document}